%杨舒云的实验报告编辑界面,使用了Huanyu Shi,2019级的模板,杨舒云在此拜谢ORZ!

%!TEX program = xelatex
\documentclass[dvipsnames, svgnames,a4paper,11pt]{article}
\input{YsySettings} 
\usepackage{lipsum}
\usepackage{adjustbox}
%\usepackage{mathrsfs} % 字体
\captionsetup[figure]{name=Figure} % 图片形式
\captionsetup[table]{name=Table} % 表格形式

\begin{document}
	
	
	
	% 实验报告封面	
	
	% 顶栏
	\begin{table}
		\renewcommand\arraystretch{1.7}
		\begin{tabularx}{\textwidth}{
				|X|X|X|X
				|X|X|X|X|}
			\hline
			\multicolumn{2}{|c|}{预习报告}&\multicolumn{2}{|c|}{实验记录}&\multicolumn{2}{|c|}{分析讨论}&\multicolumn{2}{|c|}{总成绩}\\
			\hline
			\LARGE 25 & & \LARGE25 & & \LARGE30 & & \LARGE80 & \\
			\hline
		\end{tabularx}
	\end{table}
	% ---
	
	% 信息栏
	\begin{table}
		\renewcommand\arraystretch{1.7}
		\begin{tabularx}{\textwidth}{|X|X|X|X|}
			\hline
			年级、专业: & 2022级 物理学 &组号: & 实验组2\\
			\hline
			姓名: & 戴鹏辉、杨舒云  & 学号: & 22344016、223444020\\
			\hline
			实验时间: & 2024/03/04 & 教师签名: & \\
			\hline
		\end{tabularx}
	\end{table}
	% ---
	
	% 大标题
	\begin{center}
		\LARGE 实验二 \quad 基本电路元件伏安特性的测量
	\end{center}
	% ---
	
	% 注意事项
	
	% 基本
	\textbf{【实验报告注意事项】}
	\begin{enumerate}
		\item 实验报告由三部分组成:
		\begin{enumerate}
			\item 预习报告:课前认真研读实验讲义,弄清实验原理;实验所需的仪器设备、用具及其使用、完成课前预习思考题;了解实验需要测量的物理量,并根据要求提前准备实验记录表格(可以参考实验报告模板,可以打印)。\textcolor{red}{\textbf{(20分)}}
			\item 实验记录:认真、客观记录实验条件、实验过程中的现象以及数据。实验记录请用珠笔或者钢笔书写并签名(\textcolor{red}{\textbf{用铅笔记录的被认为无效}})。\textcolor{red}{\textbf{保持原始记录,包括写错删除部分,如因误记需要修改记录,必须按规范修改。}}(不得输入电脑打印,但可扫描手记后打印扫描件);离开前请实验教师检查记录并签名。\textcolor{red}{\textbf{(30分)}}
			\item 数据处理及分析讨论:处理实验原始数据(学习仪器使用类型的实验除外),对数据的可靠性和合理性进行分析;按规范呈现数据和结果(图、表),包括数据、图表按顺序编号及其引用;分析物理现象(含回答实验思考题,写出问题思考过程,必要时按规范引用数据);最后得出结论。\textcolor{red}{\textbf{(30分)}}
		\end{enumerate}
		\textbf{实验报告就是将预习报告、实验记录、和数据处理与分析合起来,加上本页封面。\textcolor{red}{(80分)}}
		\item 每次完成实验后的一周内交\textbf{实验报告}(特殊情况不能超过两周)。
		\item \textbf{其它注意事项}:
		\begin{enumerate}
			\item 请认真查看并理解实验讲义第一章内容;
			\item 注意实验器材的合理使用;
			\item 使用结束使用各种仪器之后需要将其放回原位。
		\end{enumerate}
	\end{enumerate}
	
	% 安全
%	\textbf{【实验安全注意事项】}	
%	\begin{enumerate}
%		\item 
%	\end{enumerate}
	
	% ---
	
%	% 特别鸣谢
%	\textbf{【特别鸣谢及模板说明】}	
%	
%	感谢2019级学长石寰宇为本实验报告提供\LaTeX 模板。\textcolor{red}{\textbf{由于原实验报告模板缺少实验编号,为方便在电脑上整理,故添加自命名编号}}
%	% ---
	
	
	
	% 目录
	\clearpage
	\tableofcontents
	\clearpage
	% ---
	
	
	
	% 预习报告	
	
	% 小标题
	\setcounter{section}{0}
	\section{实验二\quad 基本电路元件伏安特性的测量 \quad\heiti 预习报告}
	% ---
	
	% 实验目的
	\subsection{实验目的}
	\begin{enumerate}
		\item 学习基本电路元件伏安特性的测试方法。
		
		\item 进一步练习直流稳压电源、万用表的使用方法。
		
	\end{enumerate}
	% ---
	
	% 仪器用具
	\subsection{仪器用具}
	\begin{table}[htbp]
		\centering
		\renewcommand\arraystretch{1.6}
		% \setlength{\tabcolsep}{10mm}
		\begin{tabular}{p{0.05\textwidth}|p{0.20\textwidth}|p{0.05\textwidth}|p{0.5\textwidth}}
			\hline
			编号& 仪器用具名称 & 数量 &  主要参数(型号,测量范围,测量精度等) \\
			\hline
			1 & 电路原理实验箱 & 1 & {\footnotesize《元件伏安特性的研究》单元和《受控源1、受控源2》单元}  \\
			\hline
			2 & 线性电阻元件   & 2 & $R_1=10K·120\Omega$ \quad $R_2=51\Omega$ \\
			\hline
			3 & 非线性电阻元件 & 1 & 12V白炽灯 \\
			\hline
			4 & 电位器$R_w$ & 1 & 可接成固定电阻、可调电阻和分压器三种形式 \\
			\hline
			5 & 直流稳压电源  & 1 &  \\
			\hline
			6 & 直流电流表、电压表  & 1 &  \\
			\hline
			7 & 直流电压表  & 1 &  \\
			\hline
			8 & 电流表专用线及实验导线  & 若干 &  \\
			\hline
		\end{tabular}
	\end{table}
	% ---
	
	% 原理概述
	\subsection{原理概述}
	\begin{enumerate}
		\item \textbf{伏安特性}:也称为电压-电流特性,是描述电路元件在不同电压作用下电流变化规律的一种特性。它反映了元件两端电压与通过元件的电流之间的关系。可以通过绘制伏安特性曲线来直观表示,其中横轴通常是电压(V),纵轴是电流(I)。
		
		如\cref{fig:fig1}所示,电路的基本元件主要包括电阻器、电容器、电感器和二极管等。每种元件的伏安特性(即电压-电流关系)都有其特点:
		
		\begin{itemize}
			\item 电阻器:遵循欧姆定律,电流与电压成正比,其比例系数即为电阻值。伏安关系线性,通过原点。
			\item 电容器:电流与电压变化率成正比,关系式为 \(I = C \frac{dV}{dt}\),其中 \(C\) 是电容值。伏安曲线表现为电压变化时电流的峰值。
			\item 电感器:电流的变化率与电压成正比,关系式为 \(V = L \frac{dI}{dt}\),其中 \(L\) 是电感值。伏安特性显示在电流变化时电压的峰值。
			\item 二极管:具有非线性伏安特性,只有当电压超过一定阈值时电流才显著增加,显示出单向导电的性质。
		\end{itemize}
		
		
		\begin{figure}[htbp]
			\centering
			\includegraphics[width=0.5\textwidth]{ET1_2Gra1.png}
			\caption{一些元件}
			\label{fig:fig1}
		\end{figure}
		
		如果将电阻元件的电压视为横坐标,电流视为纵坐标,绘制电压与电流的关系曲线,这条曲线称为该元件的伏安特性。
		
		\item \textbf{线性元件的伏安特性}:线性电阻元件的伏安特性在\(V-I\)(或\(I-V\))平面上是通过坐标原点的直线,与元件电压或电流的方向无关,是双向性的元件,最典型的线性元件是电阻器;其关系遵从欧姆定律,电阻值可由以下公式确定:\[R = \frac{V}{I}\]
		$R$表示电阻阻值,$V$表示电阻两端电压以及$I$表示流经电阻电流。
		
		\item \textbf{考虑发热对电阻伏安特性的影响}:电阻在通电时会发热,称为焦耳热。电阻的温度升高会导致其电阻值变化(对于金属电阻,温度上升,电阻值增加;对于半导体材料,温度上升,电阻值减少)。这种现象会影响电阻的伏安特性,使得原本线性的关系出现偏差。
		
		\item \textbf{常见非线性元件的伏安特性}:
		\begin{itemize}
			\item 电流控制型电阻元件:如果元件的端电压是流过该元件电流的单值函数,则称为电流控制型电阻元件。这类元件的电阻值随着通过它的电流的变化而变化,如NTC(负温度系数)热敏电阻,其电阻随温度(进而是电流)的增加而减小。
			\item 电压控制型的电阻元件:如果通过元件的电流是该元件端电压的单值函数,则称为电压控制型的电阻元件。这类元件的电阻值随着两端电压的变化而变化,如VDR(电压依赖性电阻器),其电阻随电压的增加而减小。
			\item 既是电流控制型又是电压控制型的电阻元件:如果元件的伏安特性曲线是单调增加或减少的,则该元件同时具有电流控制型和电压控制型的特性。某些特殊电阻元件可能同时受电流和电压的控制,如MEMS(微电子机械系统)技术中的某些传感器。
		\end{itemize}
		
		\begin{figure}[htbp]
			\centering
			\includegraphics[width=0.9 \textwidth]{ET1_2Gra3.png}
			\caption{非线性电阻元件的伏安特性示意图}
			\label{fig:fig3}
		\end{figure}
		
		\item \textbf{受控源}:
		\begin{itemize}
			\item 什么是受控源:受控源分为独立源(如电池和发电机)和非独立源(也称受控源),受控源的输出电压或电流随网络中某支路的电压或电流变化而变化。
			\item 四种理想受控源的转移特性表示:
			\begin{enumerate}
				\item 电压控制电压源(VCVS):输出电压与输入电压成比例,转移电压比为\(\mu\), 表示为$U_2 = \mu U_1$。
				\item 电流控制电压源(CCVS):输出电压与输入电流成比例,转移电阻为\(\gamma\), 表示为$U_2 = \gamma I_1$。
				\item 电压控制电流源(VCCS):输出电流与输入电压成比例,转移电导为\(\gamma\), 表示为$I_2 = gU_1$。
				\item 电流控制电流源(CCCS):输出电流与输入电流成比例,转移电流比为\(\beta\), 表示为$I_2 = \beta I_1$。
			\end{enumerate}
		\end{itemize}
	\end{enumerate}

		\begin{figure}[htbp]
			\centering
			\includegraphics[width=0.6 \textwidth]{ET1_2Gra4.png}
			\caption{四种理想受控源}
			\label{fig:fig4}
		\end{figure}


	% ---
	
	
	
	% 实验前思考题
	\subsection{实验预习题}
	
	% 思考题1
	\begin{question}
		预习了解电路基本元件及其伏安特性;考虑发热对电阻伏安特性的影响。
	\end{question}
	见\textbf{原理概述}部分。
	
	% 思考题2
	\begin{question}
		万用表电压档与电流档的内阻范围以及内阻对测量的影响
	\end{question}
	\begin{itemize}
		\item 电压档:万用表在测量电压时具有较高的内阻(通常在 \(M\Omega\) 级别),以减小对电路的影响。
		\item 电流档:万用表在测量电流时的内阻相对较低,为了减少测量过程中的电压降。
		\item 万用表的内阻对测量结果有一定的影响。高内阻有助于电压测量时减少对电路的加载,而低内阻有助于电流测量时减小电压降,从而提高测量准确度。
	\end{itemize}
	
	% 思考题3
	\begin{question}
		受控源和独立源相比有何异同点?比较两种受控源的代号、控制量与被控制量的关系如何?
	\end{question}
	独立源与受控源:
	\begin{itemize}
		\item 独立源:其输出不依赖于电路中其他元件的电压或电流。分为电压源和电流源。
		\item 受控源:输出依赖于电路中某些其他元件的电压或电流。分为电压控制电压源(VCVS)、电流控制电压源(CCVS)、电压控制电流源(VCCS)和电流控制电流源(CCCS)。
	\end{itemize}
	
	受控源的代号及其控制量与被控制量的关系:
	\begin{itemize}
		\item VCVS:电压控制电压源,输出电压与输入电压成正比。
		\item CCVS:电流控制电压源,输出电压与输入电流成正比。
		\item VCCS:电压控制电流源,输出电流与输入电压成正比。
		\item CCCS:电流控制电流源,输出电流与输入电流成正比。
	\end{itemize}
	
	详见\textbf{原理概述}部分。
	
	\begin{question}
		两种受控源中的g、$\gamma$的意义是什么?如何测得?
	\end{question}
	在受控源中,“g”通常表示电压控制电流源(VCCS)的转移系数,即输出电流与输入电压的比例系数。而“$\gamma$”表示电流控制电压源(CCVS)的转移系数,即输出电压与输入电流的比例系数。
	
	这些转移系数可以通过实验测定,即通过改变输入量(电压或电流),观察输出量的变化,从而计算得到。
	
	\begin{question}
		受控源输入输出是否符合能量守恒,其中的能量转移是怎么进行的?
	\end{question}
	受控源在理想情况下是符合能量守恒原则的,但它们不是能量的独立来源。受控源的输出能量来自于电路的其他部分或外部供电。在实际电路中,受控源模拟的是通过其他电路元件(如放大器)对输入信号进行放大或转换的过程,这个过程中能量的来源是这些元件的供电,而不是受控源本身。因此,受控源的能量转移实际上是通过电路中的能量转换和放大来实现的,遵循能量守恒定律。
	
	% ---
	
	
	
	% 实验记录	
	\clearpage
	
	% 顶栏
	\begin{table}
		\renewcommand\arraystretch{1.7}
		\centering
		\begin{tabularx}{\textwidth}{|X|X|X|X|}
			\hline
			专业: & 物理学 & 年级: & 2022级 \\
			\hline
			姓名: & 戴鹏辉、杨舒云 & 学号: & 22344016、22344020\\
			\hline
			室温: &  & 实验地点: & A522 \\
			\hline
			学生签名:& 见\textbf{附件}部分 & 评分: &\\
			\hline
			实验时间:& 2024// & 教师签名:&\\
			\hline
		\end{tabularx}
	\end{table}
	% ---
	
	% 小标题
	\section{ETX 实验名称×××  \quad\heiti 实验记录}
	% ---
	
	% 实验过程记录
	\subsection{实验内容、步骤与结果}
	
	%
	\subsubsection{操作步骤记录}
	\begin{enumerate}
		\item 
	\end{enumerate}	
	
	%
	\subsubsection{}
	\begin{enumerate}
		\item \begin{table}[h]
			\centering
			\caption{表格示例}
			\label{tab:tab1}
			\begin{tabular}{|c|c|c|c|c|c|}
				\hline
				组1/序号i & 1 & 2 & 3 & 4 & 5 \\
				$v_{1i}(m/s)$ & 1.26 & 1.08 & 1.00 & 0.75 & 0.38 \\
				$f_{1i}(Hz)$ & 40073 & 40127 & 40105 & 40088 & 40066 \\
				\hline
				组2/序号i & 1 & 2 & 3 & 4 & 5 \\
				$v_{2i}(m/s)$ & 1.21 & 1.06 & 0.99 & 0.52 & 0.57 \\
				$f_{2i}(Hz)$ & 40143 & 40125 & 40084 & 40080 & 40067 \\
				\hline
				组3/序号i & 1 & 2 & 3 & 4 & 5 \\
				$v_{3i}(m/s)$ & 1.15 & 0.98 & 0.78 & 0.59 & 0.36 \\
				$f_{3i}(Hz)$ & 40135 & 40115 & 40092 & 40070 & 40044 \\
				\hline
			\end{tabular}
		\end{table}		
	\end{enumerate}
	
	% ---
	
	% 原始数据
	\clearpage
	\subsection{原始数据记录}
	实验记录本上的原始数据见%\cref{}(签字)。
	
	实验台桌面整理见%\textbf{附件}部分(\cref{})。
	
	其它原始数据见%\cref{}。
	% ---
	
	% 问题记录
	\subsection{实验过程遇到问题及解决办法}
	\begin{enumerate}
		\item 
	\end{enumerate}
	% ---
	
	
	
	% 分析与讨论	
	\clearpage
	
	% 顶栏
	\begin{table}
		\renewcommand\arraystretch{1.7}
		\begin{tabularx}{\textwidth}{|X|X|X|X|}
			\hline
			专业:& 物理学 &年级:& 2022级\\
			\hline
			姓名: & 戴鹏辉、杨舒云 & 学号:& 22344016、22344020\\
			\hline
			日期:& 2024// & 评分: &\\
			\hline
		\end{tabularx}
	\end{table}
	% ---
	
	% 小标题
	\section{ETX 实验名称××× \quad\heiti 分析与讨论}
	% ---
	
	% 数据处理
	\subsection{实验数据分析}
	
	%
	\subsubsection{}
	\begin{enumerate}
		\item 
	\end{enumerate}
	
	%
	\subsubsection{}
	\begin{enumerate}
		\item 
	\end{enumerate}
	
	%
	\subsubsection{}
	
	% ---
	
	% 实验后思考题
	\subsection{实验后思考题}
	
	%思考题1
	\begin{question}
		
	\end{question}
	
	% 思考题2
	\begin{question}
		
	\end{question}
	
	% 思考题3
	\begin{question}
		
	\end{question}
	
	% ---
	
	
	% 结语部分
	\clearpage
	
	% 小标题
	\section{ETX 实验名称××× \quad\heiti 结语}
	% ---
	
	% 总结、杂谈与致谢
	\subsection{实验心得和体会、意见建议等}
	\begin{enumerate}
		\item 
	\end{enumerate}
	% ---
	
	% 参考文献
	\subsection{参考文献}
	[1] 维基百科 https://zh.wikipedia.org
	
	[2] 沈韩.基础物理实验.——北京:科学出版社,2015.2 ISBN:978-7-03-043311-4
	
	% ---
	
	% 附件
	\subsection{附件及实验相关的软硬件资料等}
	试验台桌面整理如%\cref{}所示。
	
	实验报告个人签名如\cref{fig:name}。
	
	\begin{figure}[htbp]
		\centering
		\includegraphics[width=0.7\textwidth]{name.png}
		\caption{个人签名}
		\label{fig:name}
	\end{figure}
	
	% ---
	
	相关代码已上传至Github。
	
	
	
\end{document}